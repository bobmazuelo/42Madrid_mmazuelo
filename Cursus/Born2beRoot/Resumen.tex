\documentclass[a4paper]{article}
\usepackage[spanish]{babel}
\selectlanguage{spanish}
\usepackage[utf8]{inputenc}
\usepackage[T1]{fontenc}

\usepackage[a4paper,top=3cm,bottom=1.25cm,left=1.75cm,right=1.75cm,marginparwidth=1.75cm]{geometry}

\usepackage{amsmath, amsthm, amsfonts}
\usepackage{graphicx}
\usepackage[colorinlistoftodos]{todonotes}
\usepackage[colorlinks=true, allcolors=blue]{hyperref}

\usepackage{listings}
\usepackage{xcolor}

\definecolor{codegreen}{rgb}{0,0.6,0}
\definecolor{codegray}{rgb}{0.5,0.5,0.5}
\definecolor{codepurple}{rgb}{0.58,0,0.82}
\definecolor{backcolour}{rgb}{0.95,0.95,0.92}

\lstdefinestyle{mystyle}{
    backgroundcolor=\color{backcolour},   
    commentstyle=\color{codegreen},
    keywordstyle=\color{magenta},
    numberstyle=\tiny\color{codegray},
    stringstyle=\color{codepurple},
    basicstyle=\ttfamily\footnotesize,
    breakatwhitespace=false,         
    breaklines=true,                 
    captionpos=b,                    
    keepspaces=true,                 
    numbers=left,                    
    numbersep=5pt,                  
    showspaces=false,                
    showstringspaces=false,
    showtabs=false,                  
    tabsize=2
}

\lstset{style=mystyle}

\title{Born2beRoot}
\author{Miguel Mazuelo Álvarez\\
  \small 42 Madrid\\
  \small mmazuelo@student.42madrid.com\\
  \date{}
}

\begin{document}
\maketitle

\begin{abstract}
Este documento es un ejercicio de administración de sistemas.
\end{abstract}

\section{Parte obligatoria}
\subsection{Selecci\'on del sistema operativo}
He seleccionado CentOS principalmente para entender su complejidad [...]

\subsection{Instalaci\'on}
Le he facilitado 30.8GB para hacer la parte bonus. \\
2GB de RAM es suficiente para poder correr. \\
Adaptador de red como adaptador puente para tener una ip en LAN ej (192.168.1.53) al contrario que en NAT (10.0.1.23) el cual solo el ordenador anfitrión puede conectarse. ?? \\
Se instala la versión CentOS Linux 7 \\
Se ha tratado de instalar la version CentOS Linux 8. pero está fue destruida el 31/12/2021 ya que se querían centrar en CentOS Stream 8 (esta versión carece de instalación mínima y solo existe versión escritorio).\footnote{https://www.centos.org/news-and-events/1322-october-centos-dojo-videos/}

\subsection{Configurando CentOS}
\subsubsection{Instalando dnf}
dnf es un gestor de paquetes para distribuciones basadas en RPM, es una versión mejorada de yum 
\\
Primeramente se comprueba si tenemos instalada alguna version de dnf. Por lo tanto introducimos:
\begin{lstlisting}[language=Bash]
dnf --version
\end{lstlisting}
Para instalarlo nos apoyamos con yum. Al meter como argumento -y en la instalación estamos aceptamos todo aquello que nos pregunte miestras se instala el paquete.
\begin{lstlisting}[language=Bash]
yum install dnf -y
\end{lstlisting}

\subsubsection{LVM}

\subsubsection{SELinux}
\paragraph{AppArmor}

\subsubsection{UFW}
\begin{lstlisting}[language=Bash]
dnf install epel-release -y
\end{lstlisting}
EPEL (Extra Packages for Enterprise Linux)  es un set adicional de paquetes que normalmente no está disponible en Enterprise Linux.
\begin{lstlisting}[language=Bash]
dnf install ufw -y
ufw enable
\end{lstlisting}
\begin{lstlisting}[language=Bash]
ufw status verbose
\end{lstlisting}
\paragraph{firewalld}

\subsubsection{ssh}

\subsubsection{sudo}

\section{Bonus}


\end{document}